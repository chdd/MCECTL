\documentclass[11pt]{article}
\usepackage{cite}
\usepackage{verbatim}
\usepackage{amsthm,amssymb,amsmath}

\begin{document}

\title{Computational Linguistics Assignment}
\author{Candidate 680817}
\maketitle

\section*{Question 1 (Incremental Parsing with the Perceptron Algorithm)}
\subsection*{Main features}
The paper \cite{Collins2004} describes a statistical approach to parsing
natural language sentences based on machine learning techniques (the perceptron
learning algorithm). 

As with any such approach, the method requires a preliminary stage in which the
system is trained on a set of examples. 

%TODO

In the linear model for NLP used, the
question of training equates to finding the 'best' parameter vector
$\bar{\alpha}$, where 'best' can be taken to mean the parameters which, when
used in the incremental parsing algorithm as applied to the input sentences of
the training data, will most closely produce the corresponding expected output
parses.

\footnote{
In fact, Theorem 1 of \cite{Collins2004} states only that the training method does
this when the example data is separable. However, as they note, stronger
guarantees can be obtained via \cite{Freund1999} and \cite{Collins2002}.
}

Crucially, the examples are transformed by a \emph{representation} function
$\Phi$, which aims to capture the significant features of a parse tree.

See below for a more detailed description of the training process.

\subsection*{Data Requirements}

\subsubsection*{External input}
The training data will consist of sentences paired with their corresponding
correct parse trees.

In addition to the training data, the system will clearly also need to be given
some sentences to parse.

\subsubsection*{Internal data structures}

The goal of the training stage is to produce a parameter vector containing
perceptron weights. Since the only operation required on this vector is a
scalar product, it can be simply stored as an array.

Although there is some flexibility in the ruleset used for classification, we
shall assume that this is hardcoded - however, 

\subsection*{Training}
The representation function maps from a sentence paired with a corresponding
candidate parse tree into a real vector space with dimension equal to the
number of discrete features under consideration.

The parameter vector $\bar{\alpha}$ encodes the weights of a single-layer
perceptron which is to be trained to separate correct parses for a sentence (as
seen by the representation function) from incorrect ones.

This means that $\bar{\alpha}$ will come to encode the relative significance
(positive or negative) of each feature in performing this classification.

%TODO

\subsubsection*{Averaged Perceptron Improvement}

The simplest training method involves repeatedly running the basic perceptron
algorithm on the example data either a fixed number of times or until a set of
weights is found which produces no errors on the example set.  An alternative
is to use an average of the parameter values, sampled after each example has
been processed. Intuitively, this is a good idea since it reduces the degree to
which the produced parameters depend on the ordering of the example data (which
ought not to be significant).

%TODO

\subsubsection*{Parsing}
Parsing can be seen as the problem of searching the space of possible parse
trees for a sentence to find the one which scores best according to the model
trained from the example data. This space can be very large, but the perceptron
approach allows for a heuristic search strategy. The parser proceeds
incrementally, from left to right through the sentence. After each word, the
score for each candidate parse under the perceptron model is calculated, and
the candidates can then be filtered based on these scores. 

In training we can think of the perceptron as a linear classifier, but for
parsing we are interested in more than a simple boolean decision about whether
a parse is good or not; we wish to find the \emph{best} parse for a sentence
$x$. This is in fact the parse $z$ for which the vector $\Phi(x, z)$ is closest
to being perpendicular to the hyperplane which forms the perceptron's decision
boundary. Hence we seek $\operatorname{arg max}_{z \in \textbf{GEN}(x)} \Phi(x, z) \cdot \bar{\alpha}$.

\subsubsection*{Decoding}

'Decoding' is the process of finding the parse $z$ of a sentence $x$ which maximises the value of $\Phi(x,z) \cdot \bar{\alpha}$

The filtering process after each word has been processed means that the overall
strategy amounts to a beam search: low-scoring partial hypotheses will be
dropped from consideration to ensure that space costs remain manageable.

\subsubsection*{Grammar Transformations}

Although using beam search makes the problem more tractable, there may still be
very many possible parses to consider at each iteration of the search depending
on the grammar used in the generation of candidates. The technique of
\emph{grammar transformations} is employed to ameliorate these difficulties.
 %TODO











\section*{Question 2 (Synchronous Context-free Grammars)}

\begin{verbatim}
John eats fish
John mange du poisson

Pierre does not eat fish
Pierre ne mange pas du poisson

Maria likes green apples
Maria aime les pommes vertes

James thinks that Maria eats snails
James pense que Maria mange des escargots

Proper nouns


Masculine noun
MN -> <fish, poisson>

Plural noun
NS -> <apples, pommes>
NS -> <snails, escargots>

NN -> <PN[], PN[]>
NN -> <MN[], MDET MN[]>
NN -> <NS[], PDET NS[]>

ON -> <MN[], du MN[]>

Masculine determiner
MDET -> <the, le>

Plural determiner
PDET -> <the, les>

Adjective
JJ -> <green, vertes>

JN -> <NN[], NN[]>
JN -> <JJ[] JN[], JN[] JJ[]>

Verb present tense
VPRES -> <eats, mange>
VPRES -> <likes, aime>
VPRES -> <thinks, pense>
VPRES -> <NEGVERB[], NEGVERB[]>

Verb infinitive
VINF -> <eat, mange>

Negated verb
NEGVERB -> <does not VINF[], ne VINF[] pas>

Verb phrase
VP -> <V[] that S[], V[] que S[]>
VP -> <VPRES[] JN[], VPRES[] JN[]>

TODO ?
NP -> <NN[3], NN[3]>

NOUN_PHRASE_1 -> < NOUN_M , du NOUN_M >
NOUN_PHRASE_1 -> < NOUN_PL , des NOUN_PL >

NOUN_PHRASE_2 -> < NOUN_PL, les NOUN_PL >

NOUN_PL -> < NN_PL, NN_PL >
NOUN_PL -> < JJ_PL NOUN_PL, NOUN_PL JJ_PL >

NN_PL -> < snails, escargots >
NN_PL -> < apples, pommes >

NN_M -> < fish, poisson >


VERB_PRESENT_1 -> < does not VERB_INF_1, ne VERB_INF_1 pas >

VERB_PRESENT_1 -> < eats, mange >
VERB_PRESENT_2 -> < likes, aime >

VERB_PHRASE -> < VERB_PRESENT_1[3] NP_1[4], VERB_PRESENT_1[3] NP_1[4] >
VERB_PHRASE -> < VERB_PRESENT_2[3] NP_2[4], VERB_PRESENT_2[3] NP_2[4] >

PROPER_NOUN -> < John,   John   >
PROPER_NOUN -> < Pierre, Pierre >
PROPER_NOUN -> < Maria,  Maria  >
PROPER_NOUN -> < James,  James  >
S -> < PROPER_NOUN[1] VERB_PHRASE[2], PRPOER_NOUN[1] VERB_PHRASE[2] >



\end{verbatim}
\section*{Question 3 (Sentiment Analysis and Opinion Mining)}
\subsection*{Overview}
The term 'sentiment analysis' refers to the task of automatically determining,
given a text which expresses some opinion on some subject, to what degree that
opinion can be considered positive or negative. This concept can be applied on
the level of whole documents or to individual sentences, and naturally the two
problems are closely related. The following explanations focus more on the
problem at the document scope, since that is the problem that is most relevant
for assessing the favourability or otherwise of online product reviews.

%TODO Opinion Mining

\subsection*{Methods}

\subsubsection*{`Bag of words'}
Perhaps the most obvious technique that is applicable is that of the `bag of
words', in which the words contained within the document are considered without
regard to their ordering; instead it is their frequency which is used to
determine a result. The system will include a lookup table which lists a
`polarity' value for some set of significant words, which value represents the
degree to which the corresponding word should be considered positive or
negative.

For instance, in a system where $1.0$ is maximally positive and
$-1.0$ is maximally negative, the word `good' might be assigned the value $0.9$
while neutral words like `the' would be $0.0$. This table could be constructed
manually, but a better approach would be to learn it from a set of example
documents whose correct sentiment is already known. Word-polarity learning is a
more widely applicable problem, and is discussed later.

Assuming the word polarities are already available, it is simple to combine these with the word frequencies in the document under consideration to produce an overall sentiment value.

Formally, if $W$ is the set of words in the document, $f:W \to \mathbb{N}$ maps
words to their frequencies, and $\Phi:W \to [-1, 1]$ their polarities, then the
sentiment value is given by $\alpha = \sum_{w \in W} f(w)\cdot\Phi(w)$.

Note that this can be considered as a scalar product in a real vector space, and is in fact analagous to the linear NLP model used in question 1.

The main advantage of the bag-of-words is its simplicity - while it can be a
good starting point for building more advanced techniques for sentiment
analysis, it is rather naïve by itself. Since words are considered
individually, the phrase `not good at all' might falsely be taken to reflect
positive sentiment, for instance.

\subsubsection*{Learning word polarities}


\bibliography{Refs}{}
\bibliographystyle{plain}
\end{document}
