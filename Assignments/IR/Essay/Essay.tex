\documentclass{acm_proc_article-sp}
\usepackage{cite}

\begin{document}

\title{Information Retrieval Assignment: Question 7}

%\numberofauthors{1} 
\author{
%\alignauthor
Candidate 680817
}

\maketitle
\begin{abstract}
This paper is a concise survey of techniques for image edge detection. A number
of different methods are described, followed by a look at some of the current
research directions, and a more detailed review of Canny's ever-popular
approach to the problem.
\end{abstract}

\section{Introduction}

Typically digital images consist of grids of pixel data without any
accompanying clues as to the semantic content. An information retrieval system
which accepts images as documents must therefore attempt to extract some level
of meaning from the given pixel data. Broadly, this problem is a subdomain of
the field of computer vision, and a wide variety of techniques have been
developed to tackle it. 

A common initial step is to apply a simplifying transformation to the data,
preferably without removing any information that is useful for the purposes of
information retrieval. Edge detection is an important example of this kind of
technique. The aim is to isolate the boundaries of the various discrepant parts
of a scene, typically as a precursor to more sophisticated image segmentation
or object detection processes.

Part of the difficulty involved in creating a good edge detector is due to the
fact that it is not immediately obvious what properties a good edge detector
should have. Although much progress has been made towards a consensus on this,
the requirements are still to some extent dictated by the specific application.
For that reason, and in the absence of any single perfect solution, the
following section will introduce (in approximate order of development) a number
of the more significant techniques and concepts that have been brought to bear
on the problem.

\section{Background}

The first significant work on edge detection was published in the early 1970s,
with the paper presenting Sobel's method \cite{Sobel:1970:CMM:905376} --
perhaps still one of the best known approaches -- being one of the earliest
contributions. As with much of the work that followed, this
approach is based on \textit{convolving} the image with a small `kernel' or
`mask' matrix, which is designed so that the result at each pixel is an
approximation of the gradient of the image intensity. 

Convolution is a powerful operation which, in its general form, is central to
functional analysis and signal processing of all kinds. Still, as applied to 2D
images the concept is simple enough -- even if the implications are not. For each
pixel $p$ in the source image, the corresponding value in the output is
calculated by centering the kernel matrix on $p$, multiplying the pairs of
thusly coincident values, and summing the results.

Sobel used two separate kernels -- $S_H$ and $S_V$ approximate the horizontal
and vertical partial derivatives respectively:
\begin{displaymath}
S_H = \begin{bmatrix} 
-1 & 0 & +1 \\
-2 & 0 & +2 \\
-1 & 0 & +1 
\end{bmatrix}
\quad
S_V = \begin{bmatrix} 
-1 & -2 & -1  \\
\ \ 0 & \ \ 0 & \ \ 0 \\
+1 & +2 & +1 
\end{bmatrix}
\end{displaymath}

After obtaining the results of both convolutions, the values are combined (via
the $L^1$ or $L^2$ norm) to give a crude approximation of the gradient. If the
magnitude of the gradient exceeds a given threshold, the corresponding pixel
can be deemed to form part of an edge.

% prewitt, kirsch etc
Prewitt's operator\cite{prewitt1970object} works in a very similar way, but
uses a different kernel. It is difficult to evaluate the effectiveness of these
techniques objectively without any clearly defined mathematical model of what
an edge actually \emph{is} -- no such model was initially available, and a
number of alternative kernels are available which have various subtly different
performance characteristics. The Kirsch operator\cite{kirsch1971computer} attempts to detect diagonal
edges as well as horizontal and vertical ones, for example.

% non-maximum supression

These classical methods are simple to implement and can be very efficient.  The
time complexity is $\mathcal{O}(mn)$ where $m$ is the number of pixels in the
input image and $n$ is the number of pixels in the kernel matrix, and since
only local data is used, the computation can easily be parallelised. The
advantage of good performance should not be underestimated in this problem;
edge detection algorithms may commonly be applied to very large amounts of data
-- perhaps from many frames of video, for example -- so very computationally
intensive approaches are often impractical even if they produce higher quality 
results.

However, by themselves, these operators are (for many purposes) overly
simplistic. They do not, in themselves, apply any kind of smoothing to the
input image -- real world images are frequently noisy, and unless this is
accounted for, the edge detector is likely to produce many false positives.

% Marr+ Hildreth, gaussian
A subsequent set of approaches, pioneered by Marr \&
Hildreth\cite{marr1980theory}, were based on Gaussian filters.  The basic
technique, known as LOG, is to find the zero-crossings of the Laplacian, after
applying a Gaussian filter of an appropriate size. The introduction of a
filtering stage was an important improvement since it reduced the negative
effects of noise in the image significantly. However, the simple thresholding
applied to determine edgedom meant that results still suffered from edge
streaking (caused when a region of gradients straddles the threshold value).

% Canny
The next major development -- and perhaps \emph{the} major paper on edge
detection -- was Canny's detector\cite{4767851}. The first achievement of
Canny's work was to explicitly develop a precise mathematical description of
the problem of edge detection and the types of property that putative solutions
should possess. The second was to use this model to successfully derive an
improved algorithm for edge detection. The paper analysis below focuses on
Canny's approach in more detail, so for now a brief outline of the improvements
will suffice.

The method uses an initial smoothing filter to reduce unwanted effects of
noise, followed by a standard discrete differentiation operator (such as Sobel)
to produce a gradient approximation. The main innovations are the stages
applied after this: \emph{adaptive thresholding} and \emph{hysteresis}.

Rather than a single threshold for edge presence, two are used: if the higher
threshold is exceeded, the pixel is marked as a `strong' edge, and if only the
lower is, the pixel is counted as a `weak' edge. The hysteresis procedure then
reexamines the weak edge pixels, and if they are sufficiently connected to the
strong ones then they are promoted; if they are not they are dropped.

Canny's improvements result in a significantly more robust method for edge
detection, and his algorithm is still a popular choice for many applications.
It is not perfect; according to Basu\cite{1097737}, there are situations where
it still suffers from false edges in the presence of reasonably small amounts
of noise.  However, if a linear solution is required, it is difficult to beat.

Another difficulty is choosing appropriate values for the threshold parameters,
as well as for the size of the smoothing mask. This must be done carefully to
have any hope of obtaining useful results, and one might argue that a more
sophisticated algorithm ought not to require this manual fine-tuning.

A deeper problem, which Canny's method also does not account for directly, is
that the edges in an image are often not the ideal, thin lines that these
techniques are good at finding. For the purposes of object detection or image
segmentation, it may be desirable to interpret a line of several pixels width
as a single edge -- the edge detection operators described hitherto will
generally take this to be two or more parallel edges, instead.

%scale space
`Multi-resolution' methods set out to handle this problem by considering the
image at a variety of scales. A key concept is that of an image's \emph{scale
space}: this space consists of versions of the image to which increasingly wide
Gaussian filters have been applied. The more filtered versions can then be
analysed to find the broader edges. These techniques vary in the range of
scales used and, more importantly, in the way they combine the results of the
different scales.

\section{State of the art}

More recently, a wide selection of more exotic approaches have been explored,
with varying degrees of success.

% Phase congruence
Borrowing from Fourier analysis, the \emph{phase congruency}
technique\cite{Kovesi} involves examining the frequency components of the
image. This information can be used for edge detection since the components of
the edges tend to be in the same phase. Phase congruency has the advantage of
performing well under varied lighting and contrast conditions, but is
significantly more computationally expensive.

% anisotropic diffusion
Another interesting non-linear method is based on \emph{anisotropic
diffusion}\cite{perona1990scale}. This is related to the heat equation (from
physics); image intensity corresponds to temperature, and diffusion over time
produces increasingly blurred versions of the image: the scale space! It is
possible to apply the equations backwards to obtain a method for edge
detection. However, the standard heat equation is direction invariant: better
results are achieved by varying the diffusion parameters over space as well as
time -- this is the anisotropic version of the technique.

% machine learning
As with many of the more non-trivial tasks in computer science, efforts have
been made to apply the techniques of machine learning to edge detection. Fuzzy
neural networks\cite{Lu20032395}, Support Vector Machines\cite{Zheng20041143},
and genetic algorithms\cite{Bhandarkar19941159} have all been used with some
success. Although it is generally not possible to analytically confirm that
these techniques provide a canonical, correct solution, the results are
nevertheless very promising and have the potential to be faster than Canny's
method.

% SUSAN
One very practically efficacious method is known (rather affably) as
SUSAN\cite{smith1997susan}. The technique involves positioning a circular mask
over each pixel, and comparing the intensity values of the pixels enclosed
using a particular empirically designed function. The algorithm can be
implemented efficiently and produces better results than Canny in corner and
junction areas.\footnote{However, please note that SUSAN is patented in the
UK.}

% wavelets
Finally, an important direction for methods taking advantage of the scale-space
representation is the use of wavelets.\cite{heric2007combined} While a full
explication of the mathematics involved in wavelet methods is beyond the scope
of this paper -- in brief, the concept allows for operations similar to the
Fourier transform but using a different set of orthogonal basis functions.
These functions are known as wavelets.  Certain wavelet families (such as the
Haar wavelet) have properties which may make them better suited to image
processing.
\begin{figure}[h]
\begin{displaymath}
\psi(t) = \begin{cases}1 
\quad & 0 \leq  t < 1/2,\\
 -1 & 1/2 \leq t < 1,\\0 &\mbox{otherwise.}\end{cases}
\end{displaymath}
\caption{Haar wavelet}
\end{figure}
In a pleasantly cyclic correspondence, the Haar mother wavelet can almost be
thought of as a functional analog of the original Prewitt
operator!\footnote{Although the way it is applied is, of course, much more
sophisticated.}

This summary is not exhaustive, but provides a taste of some of the areas into
which modern edge detection methods are progressing. For a more detailed
survey, please see Oskoei \& Hu\cite{oskoei2010survey}.

\section{Paper analysis}

In the following section we take a more detailed look at Canny's original
paper\cite{4767851} and the detection method described therein, and review the
content with some consideration to its relevancy today.

Canny's `computational approach' to edge detection set the field on a more
solid theoretical basis: in contrast to the generally rather subjective
approaches which had preceded, he began by formulating a fixed set of
performance criteria: high signal-to-noise ratio, accurate location of the
found edges, and the requirement that the same edge should not be detected
repeatedly.

He then successfully constructed a mathematical model for edge detection in
which these properties could be expressed. This provides a way to find optimal
detectors for various types of edges by numerically optimising a function which
captures the desired properties.

The model chosen for use in the rest of the paper is based on the conception of
the ideal edge as a step function. The optimal detector for these edges as
derived from the model is shown to be superior to the more standard first
derivative of Gaussian operator.

Some effort is made to allow for determining suitable thresholding parameters
by estimating noise levels in the source image; however, depending on the
characteristics of the edges in the image, manual adjustment may be necessary.

With this foundation in place, Canny extends the techniques thus far developed
into the second dimension, by showing how to combine results from multiple
independent directions in the manner previously described.  The final section
of the paper is concerned with computationally justifying this decision: he
shows mathematically that multiple directional operators are necessary for the
accurate detection of all orientations of edges.

Some attention is also paid to an approach for combining detection results from
analysis of the same image at a range of resolution levels -- he refers to this
as `feature synthesis'. While this aspect of Canny's detector has been
superseded, the ideas presage the \emph{scale space} concept by some years and
led the way for further exploration.

 %
 %conclusion
To conclude, we consider the importance of the paper both historically and as a
description of a method that is still widely used today. In the first place,
``A Computational Approach'' is undeniably (through citation counts if nothing
else) a landmark paper in information retrieval. There are two separate reasons
for this: firstly, the method itself is remarkably effective; not only did it
represent a significant step up in accuracy compared to the best methods of the
day (whilst retaining computational efficiency and a reasonable level of
simplicity), but it also remained the method of choice for practical
implementations for many years following publication. Secondly, the rigorous
analytical style of treatment introduced by Canny laid solid foundations for a
more focused research attitude in the problem area.

As a modern method, Canny's detector is finally beginning to be eclipsed, at
least in accuracy, by more advanced approaches such as SUSAN and wavelet
transforms; yet for everyday use it remains relevant.\footnote{In fact, thanks
to some fortuitous commissioning and scheduling,
 even the Dalek images Canny used for samples are still topical!} For demanding
 problems like image segmentation, a very high quality of detection is
 required, and in these cases there is still a need for the newest techniques.
 But in most situations, Canny's method is perfectly sufficient, and with a
 wide range of implementations available (even on the
 GPU\cite{10.1109/CVPRW.2008.4563088}) it is certainly still worthy of
 consideration whenever edges need to be detected.
 

\bibliographystyle{abbrv}
\bibliography{Refs}{}



\balancecolumns
% That's all folks!
\end{document}
