\documentclass{acm_proc_article-sp}
\usepackage{cite}

\begin{document}

\title{Information Retrieval Assignment: Question 7}

%\numberofauthors{1} 
\author{
%\alignauthor
Candidate 680817
}

\maketitle
\begin{abstract}
This paper is a concise survey of methods for image edge detection. A number of
different techniques are described, followed by a look at some of the
directions that current methods are investigating, and a more detailed review
of Canny's ever-popular approach to the problem.
\end{abstract}

\section*{Introduction}

Typically digital images consist of grids of pixel data without any
accompanying clues as to the semantic content. An information retrieval system
which accepts images as documents must therefore be able to extract some level
of meaning from the given pixel data. Broadly, this problem is a subdomain of
the field of Computer Vision, and a wide variety of techniques have been
developed to tackle it. The techniques used to initially transform the data aim
to simplify the image without removing any information that is useful for the
purposes of information retrieval. Such techniques include Fourier Analysis,
\textbf{TODO} and Edge Detection. While each of these is valuable in its own right, and
an effective system may well combine multiple techniques for more accurate
results, this paper will focus only on methods for Edge Detection.



\section*{Background}

The first significant work on edge detection was published in the early 1970s,
with the paper presenting Sobel's method \cite{Sobel:1970:CMM:905376} --
perhaps still one of the best known approaches -- being one of the most
important early contributions. As with much of the work that followed, this
approach is based on \textit{convolving} the image with a small `kernel' or
`mask' matrix which is designed so that the result at each pixel is an
approximation of the gradient of the image intensity. 

Convolution is a powerful operation which, in its general form, is central to
functional analysis and signal processing of all kinds. Still, as applied to 2D
images the concept is simple enough -- even if the implications are not. For each
pixel $p$ in the source image, the corresponding value in the output is
calculated by centering the kernel matrix on $p$, multiplying the pairs of
thusly coincident values, and summing the results.

Sobel used two separate kernels -- $S_H$ and $S_V$ detect horizontal and vertical edges respectively:
\begin{displaymath}
S_H = \begin{bmatrix} 
-1 & 0 & +1 \\
-2 & 0 & +2 \\
-1 & 0 & +1 
\end{bmatrix}
\quad
S_V = \begin{bmatrix} 
-1 & -2 & -1  \\
\ \ 0 & \ \ 0 & \ \ 0 \\
+1 & +2 & +1 
\end{bmatrix}
\end{displaymath}

After obtaining the results of both convolutions, the values are combined to
give a crude approximation of the gradient. If the magnitude of the gradient
exceeds a given threshold, the corresponding pixel can be deemed to form part
of an edge.

% prewitt, kirsch etc
Prewitt's operator\cite{prewitt1970object} works in a very similar way, but
uses a different kernel. It is difficult to evaluate the effectiveness of these
techniques objectively without any clearly defined mathematical model of what
an edge actually \emph{is} -- no such model was initially available, and a
number of alternative kernels are available which have different performance
characteristics depending on the properties of the source image. The Kirsch
operator attempts to detect diagonal edges as well as horizontal and vertical
ones, for example.

% non-maximum supression

These classical methods are simple to implement and can be very efficient.  The
time complexity is $\mathcal{O}(mn)$ where $m$ is the number of pixels in the
input image and $n$ is the number of pixels in the kernel matrix, and since
only local data is used, the computation can easily be parallelised. The
advantage of good performance should not be underestimated in this problem;
edge detection algorithms may commonly be applied to very large amounts of data
-- perhaps from many frames of video, for example -- so very computationally
intensive approaches are often impractical even if they produce higher quality 
results.

However, by themselves, these operators are (for many purposes) overly
simplistic. They do not, in themselves, apply any kind of smoothing to the
input image -- real world images are frequently noisy, and unless this is
accounted for, the edge detector is likely to produce many false positives.

% Marr+ Hildreth, gaussian
% LOG !!

% Canny
The next major development -- and perhaps \emph{the} major paper on edge
detection -- was Canny's detector\cite{4767851}. The first achievement of Canny's work was to
explicitly develop a precise mathematical description of the problem of edge
detection and the types of property that putative solutions should possess. The
second was to use this model to successfully derive an improved algorithm for
edge detection. The paper analysis below focuses on Canny's approach in more
detail, so for now a brief outline of the improvements will suffice.

% canny TODO **

Canny's improvements result in a significantly more robust method for edge
detection, and his algorithm is still a popular choice for many applications.
It is not perfect; according to \cite{1097737}, there are situations where it
still suffers from false edges in the presence of reasonably small amounts of
noise.  However, if a linear solution is required, it is difficult to beat.

Another difficulty is choosing appropriate values for the threshold parameters,
as well as for the size of the smoothing mask. This must be done carefully to
have any hope of obtaining useful results, and one might argue that a more
sophisticated algorithm ought not to require this manual fine-tuning.

A deeper problem, which Canny's method also does not account for directly, is
that the edges in an image are often not the ideal, thin lines that these
techniques are good at finding. For the purposes of object detection or image
segmentation, it may be desirable to interpret a line of several pixels width
as a single edge -- the edge detection operators described hitherto will
generally take this to be two or more parallel edges, instead.

%scale space
`Multi-resolution' methods set out to handle this problem by considering the
image at a variety of scales. A key concept is that of an image's \emph{scale
space}: this space consists of versions of the image to which have increasingly
large Gaussian filters have been applied. Performing edge detection on the more
filtered versions will then find the broader edges. These techniques vary in
the range of scales used and the way they combine the results of the different
scales.

\section*{State of the art}

More recently, a wide selection of more exotic approaches have been explored,
with varying degrees of success.

% Phase congruence
Borrowing from Fourier analysis, the \emph{phase congruency}
technique\cite{Kovesi} involves examining the frequency components of the
image. This information can be used for edge detection since the components of
the edges tend to be in the same phase. Phase congruency has the advantage of
performing well under varied lighting and contrast conditions, but is
significantly more computationally expensive.

% anisotropic diffusion
Another interesting non-linear method is based on \emph{anisotropic diffusion}


% machine learning
As with many of the more non-trivial tasks in computer science, efforts have
been made to apply the techniques of machine learning to edge detection. Fuzzy
neural networks\cite{Lu20032395}, Support Vector Machines\cite{Zheng20041143},
and genetic algorithms\cite{Bhandarkar19941159} have all been used with some
success. Although it is generally not possible to analytically confirm that
these techniques provide a canonical, correct solution, the results are
nevertheless very promising and have the potential to be faster than Canny's
method.

% SUSAN
Another practically efficacious method is known (rather affably) as SUSAN


% wavelets
Finally, 

This summary is not exhaustive, but provides a taste of some of the areas into
which modern edge detection methods are progressing. For a more detailed
survey, please see Oskoei & Hu\cite{oskoei2010survey}.

\section*{Paper analysis}

In the following section we take a more detailed look at Canny's original paper\cite{4767851} and the detection method described therein.

% canny ?
 % theory
 % gpu ?



\bibliographystyle{abbrv}
\bibliography{Refs}{}



\balancecolumns
% That's all folks!
\end{document}
